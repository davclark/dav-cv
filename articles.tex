\begin{rubric}{Publications}

\text{\textbf{Clark, D.}, Ranney, M. A., \& Felipe, J., (2013, in press).
    Knowledge Helps: Mechanistic Information and Numeric Evidence as Cognitive
    Levers to Overcome Stasis and Build Public Consensus on Climate Change.
    \emph{Proceedings of the 35th Annual Conference of the Cognitive Science
    Society}. }

\text{}

\text{Ranney, M. A., \textbf{Clark, D.}, Reinholz, D. L., \& Cohen, S. (2012).
    Changing Global Warming Beliefs with Scientific Information: Knowledge,
    Attitudes, and RTMD (Reinforced Theistic Manifest Destiny Theory). In N.
    Miyake, D. Peebles, \& R.P. Cooper (Eds.), \emph{Proceedings of the 34th
    Annual Conference of the Cognitive Science Society} (pp. 2228-2233). Austin,
    TX: Cognitive Science Society.}

\text{}

\text{\textbf{Clark, D.}, Reinholz, D. L., Cohen, S., \& Ranney, M. A. (2012).
    Improving Americans’ Modest Global Warming Knowledge in the Light of RTMD
    (Reinforced Theistic Manifest Destiny) Theory. In J. van Aalst, K. Thompson,
    M. M. Jacobson, \& P. Reimann (Eds.), \emph{The Future of Learning:
    Proceedings of the Tenth International Conference of the Learning Sciences,}
    Volume 2 (pp. 2-481 to 2-482).  International Society of the Learning
    Sciences, Inc.}

\text{}

\text{Gorgolewski, K., Burns, C. D., Madison, C., \textbf{Clark, D.}, Halchenko, Y.
    O., Waskom, M. L., \& Ghosh, S. S. (2011). Nipype: a flexible, lightweight
    and extensible neuroimaging data processing framework in python. \emph{Frontiers
    in Neuroinformatics, 5,} 13. doi:10.3389/fninf.2011.00013}

\text{}

\text{\textbf{Clark, D.}, \& Ranney, M. A. (2010). Known knowns and unknown
knowns: Multiple memory routes to improved numerical estimation. In K.
Gomez, L. Lyons, \& J.  Randinsky (Eds.), \emph{Learning in the Disciplines:
Proceedings of the Ninth International Conference of the Learning Sciences, Vol.
1-Full Papers (pp.  460-467).} International Society of the Learning Sciences,
Inc.}

\text{}

\text{\textbf{Clark, D.}, \& Ivry, R. B. (2010). Multiple systems for motor
skill learning. \emph{Wiley Interdisciplinary Reviews: Cognitive Science, 1}(4),
461-467. doi:10.1002/wcs.56}

\text{}

\text{Hasson, U., Furman, O., \textbf{Clark, D.}, Dudai, Y., and Davachi, L. (2008). Enhanced
intersubject correlations during movie viewing correlate with successful
episodic encoding. \emph{Neuron, 57}(3), 452-462.
[Note: First 3 authors contributed equally]
% (http://www.psych.nyu.edu/davachilab/docs/Neuron-2008-Hasson.pdf)
}

\text{}

\text{Kahn, I., Pascual-Leone, A., Theoret, H., Fregni, F., \textbf{Clark, D.}, \&
Wagner, A.  D.  (2005). Transient disruption of ventrolateral prefrontal cortex
during verbal encoding affects subsequent memory performance. \emph{Journal of
Neurophysiology, 94,} 688-698.
% (http://www-psych.stanford.edu/~wagner/Publications/papers/KAHN_JNP05.pdf)
}

\text{}

\text{\textbf{Clark, D.}, \& Wagner, A. D. (2003).  Assembling and encoding 
word representations: fMRI subsequent memory effects implicate a role for
phonological control. \emph{Neuropsychologia, 41,} 304-317.
% (http://web.mit.edu/wagner/www/papers/CLA_NEUROP03.pdf)
}

\end{rubric}
