\begin{rubric}{Publications}

\text{Ranney, M. A., \& \textbf{Clark, D.} (2016). Climate Change Conceptual
    Change: Scientific Information Can Transform Attitudes. \emph{Topics in
    Cognitive Science 8}, 1, 49--75. \emph{doi:10.1111/tops.12187}}

\text{}

\text{\textbf{Clark, D.}, Schumann, F., \& Mostofsky, S. H. (2015). Mindful
    movement and skilled attention. \emph{Frontiers in Human Neuroscience, 9},
    297.  \emph{doi:10.3389/fnhum.2015.00297} }

\text{}

\text{\textbf{Clark, D.}, (2014). MTurk Admin. Retrieved from Open Science
    Framework. \emph{osf.io/iwdru}}

\text{}

\text{Gorgolewski, K., Burns, C. D., Madison, C., \textbf{Clark, D.},
Halchenko, Y. O., Waskom, M. L., \& Ghosh, S. S. (2011). Nipype: a flexible,
lightweight and extensible neuroimaging data processing framework in python.
\emph{Frontiers in Neuroinformatics, 5,} 13.
\emph{doi:10.3389/fninf.2011.00013}}

\text{}

\text{\textbf{Clark, D.}, \& Ivry, R. B. (2010). Multiple systems for motor
skill learning. \emph{Wiley Interdisciplinary Reviews: Cognitive Science, 1}(4),
461--467. \emph{doi:10.1002/wcs.56}}

\text{}

\text{Hasson, U., Furman, O., \textbf{Clark, D.}, Dudai, Y., \& Davachi, L.
    (2008). Enhanced Intersubject correlations during movie viewing correlate
    with successful episodic encoding. \emph{Neuron, 57}(3), 452--462.  [Note:
    First 3 authors contributed equally]
    % (http://www.psych.nyu.edu/davachilab/docs/Neuron-2008-Hasson.pdf)
}

\text{}

\text{Kahn, I., Pascual-Leone, A., Theoret, H., Fregni, F., \textbf{Clark, D.},
    \& Wagner, A.  D.  (2005). Transient disruption of ventrolateral prefrontal
    cortex during verbal encoding affects subsequent memory performance.
    \emph{Journal of Neurophysiology, 94,} 688--698.
    % (http://www-psych.stanford.edu/~wagner/Publications/papers/KAHN_JNP05.pdf)
}

\text{}

\text{\textbf{Clark, D.}, \& Wagner, A. D. (2003).  Assembling and encoding
    word representations: fMRI subsequent memory effects implicate a role for
    phonological control. \emph{Neuropsychologia, 41,} 304--317.
    % (http://web.mit.edu/wagner/www/papers/CLA_NEUROP03.pdf)
}

\end{rubric}

\begin{rubric}{Peer-reviewed Conference Papers}

% https://library.iated.org/view/CLARK2017WHA

\text{\textbf{Clark, D.}, Okamoto, Y., Chiba, S. (2017). What drives student
motivation? An intensive, tablet-based, experience sampling approach.
\emph{EDULEARN17 Proceedings.} Barcelona. \emph{10.21125/edulearn.2017.1290}}

\text{}

\text{Turek, D., Suen, C., \& \textbf{Clark, D.} (2015). Pushing Data Science
    Education into the Real World. Presented at the Bloomberg Data for Good
    Exchange Conference, New York, NY.}

\text{}

\text{\textbf{Clark, D.}, Culich, A., \& Hamlin, B. (2014). A Common Scientific Compute
    Environment for Research and Education. \emph{Proceedings of SciPy, 2014.}
    Austin, TX.}

\text{}

\text{\textbf{Clark, D.}, Ranney, M. A., \& Felipe, J., (2013).
    Knowledge Helps: Mechanistic Information and Numeric Evidence as Cognitive
    Levers to Overcome Stasis and Build Public Consensus on Climate Change.
    In M. Knauff, M. Pauen, N. Sebanz, \& I. Wachsmuth (Eds.),
    \emph{Cooperative Minds: Social Interaction and Group Dynamics; Proceedings
        of the 35th Annual Meeting of the Cognitive Science Society}
    (pp. 2070--2075). Austin, TX: Cognitive Science Society.  }

\text{}

\text{Ranney, M. A., \textbf{Clark, D.}, Reinholz, D. L., \& Cohen, S. (2012).
    Changing Global Warming Beliefs with Scientific Information: Knowledge,
    Attitudes, and RTMD (Reinforced Theistic Manifest Destiny Theory). In N.
    Miyake, D. Peebles, \& R.P. Cooper (Eds.), \emph{Proceedings of the 34th
    Annual Conference of the Cognitive Science Society} (pp. 2228--2233). Austin,
    TX: Cognitive Science Society.}

\text{}

\text{\textbf{Clark, D.}, Reinholz, D. L., Cohen, S., \& Ranney, M. A. (2012).
    Improving Americans’ Modest Global Warming Knowledge in the Light of RTMD
    (Reinforced Theistic Manifest Destiny) Theory. In J. van Aalst, K. Thompson,
    M. M. Jacobson, \& P. Reimann (Eds.), \emph{The Future of Learning:
    Proceedings of the Tenth International Conference of the Learning Sciences,}
    Volume 2 (pp. 2-481 to 2-482).  International Society of the Learning
    Sciences, Inc.}

\text{}

\text{\textbf{Clark, D.}, \& Ranney, M. A. (2010). Known knowns and unknown
    knowns: Multiple memory routes to improved numerical estimation. In K.
    Gomez, L. Lyons, \& J.  Randinsky (Eds.), \emph{Learning in the Disciplines:
        Proceedings of the Ninth International Conference of the Learning
        Sciences, Vol.  1-Full Papers (pp. 460--467).} International Society of
    the Learning Sciences, Inc.}

\end{rubric}


\begin{rubric}{Retraction}

    \text{Turek, D., Suen, A., and \textbf{Clark, D.} (2016--retracted). The
        BIDS Collaborative: Towards Effective Graduate Student Data Science
        Projects.}

\end{rubric}
